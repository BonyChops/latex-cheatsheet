\documentclass[a4j, titlepage]{jarticle}
\usepackage{float}
\usepackage[dvipdfmx]{graphicx}
%\usepackage{mediabb}
\makeatletter
%https://qiita.com/ta_b0_/items/2619d5927492edbb5b03
\usepackage{listings,jlisting} %日本語のコメントアウトをする場合jlstlistingが必要
\usepackage[top=30truemm,bottom=30truemm,left=25truemm,right=25truemm]{geometry}
%ここからソースコードの表示に関する設定
\lstset{
    basicstyle={\ttfamily},
    identifierstyle={\small},
    commentstyle={\smallitshape},
    keywordstyle={\small\bfseries},
    ndkeywordstyle={\small},
    stringstyle={\small\ttfamily},
    frame={tb},
    breaklines=true,
    columns=[l]{fullflexible},
    numbers=left,
    xrightmargin=0zw,
    xleftmargin=3zw,
    numberstyle={\scriptsize},
    stepnumber=1,
    numbersep=1zw,
    lineskip=-0.5ex
}
%ここまでソースコードの表示に関する設定

\if0
---------------------------------こめこめこめこめこめこめこめこめこめこめこめ
\renewcommand{\thefigure}{\arabic{figure}}
\@addtoreset{figure}{section}
\makeatother
\makeatletter
\renewcommand{\thetable}{\arabic{table}}
\@addtoreset{table}{section}
\makeatother
\makeatletter
\renewcommand{\theequation}{\arabic{equation}}
\@addtoreset{equation}{section}
\makeatother
---------------------------------こめこめこめこめこめこめこめこめこめこめこめ
\fi

\usepackage{multicol}

\usepackage{amssymb}
\usepackage{amsmath}
\usepackage{url}
\usepackage{ascmac}
\usepackage{fancyvrb}
\usepackage{otf}
\usepackage{here}


\title{\LaTeX の俺的最強チートシート}
\author{Bony\_Chops}

\begin{document}
\maketitle

\setcounter{section}{-1}
\section{はじめに}
\textbf{Hello hackers!} これから\textbf{Word}から\LaTeX を使い始めるキミや、ある程度\LaTeX には慣れたけどコマンドを忘れがちなキミに向けたチートシートです。
\subsection{環境}
本チートシートは以下の環境を使っていることを想定して書いています。特殊な仕様意外は基本的に同じだと思います。

\begin{table*}[htbp]
    \center
    \caption{筆者の環境}
    \begin{tabular}{|c|c|} \hline
        OS & Ubuntu 20.04 \\ \hline
        エディタ & Visual Studio Code \\ \hline
        環境 & TeX Live 2017 \\ \hline
    \end{tabular}
\end{table*}

\section{コマンド集}

\begin{itembox}[l]{\textbackslash section $\cdot$ \textbackslash subsection $\cdot$ \textbackslash subsubsection}

\begin{multicols}{2}

\subsubsection*{コマンド}

\begin{lstlisting}
\section{目的}
今回はりんごの剥き方を理解することを
目的とする。
\end{lstlisting}

\newpage
\subsubsection*{実行結果}
\setcounter{section}{0}
\begin{shadebox}
\section{目的}
今回はりんごの剥き方を理解することを目的とする。
\end{shadebox}



\end{multicols}

\end{itembox}



\begin{thebibliography}{9}
    \bibitem{text} 令和2年度後期工学実験実習IIIテキスト
\end{thebibliography}


\end{document}